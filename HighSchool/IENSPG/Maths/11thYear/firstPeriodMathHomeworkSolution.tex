\documentclass[12pt, letterpaper, twoside]{article}
\usepackage[utf8]{inputenc}
\usepackage{geometry}
 \geometry{
 a4paper,
 total={170mm,257mm},
 left=20mm,
 top=20mm,
 }
\title{Matemáticas $1^{er}$ periodo}
\author{Juan Manuel Young Hoyos}
\date{March 17, 2021}

\begin{document}

\maketitle

\section*{ACTIVIDAD 1: DESIGUALDADES}

Realice los ejemplos que se dejaron propuestos en las propiedades de las desigualdades.

\section{Transitividad}

$a, b, c \in R, a < b \;y\; b < c \to a < c$

\begin{itemize}
    \item $a = 1, b = 2, c = 3 \in R, 1 < 2\;y\;2 < 3 \to 1 < 3$
    \item $a = 3, b = 5, c = 8 \in R, 3 < 5\;y\;5 < 8 \to 3 < 8$
    \item $a = 3, b = 2, c = 1 \in R, 3 > 2\;y\;2 > 1 \to 3 > 1$
\end{itemize}

\section{Adición y sustracción}

$a, b, c \in R, a < b \to a \pm c < b \pm c$
\begin{itemize}
    \item $a = 1, b = 2, c = 3 \in R, 1 < 2 \to 1 \pm 3 < 2 \pm 3$
   
    Sumando, $ 4 < 5$
    
    Restando, $ -2 < -1$

    \item $a = 5, b = 35, c = 23 \in R, 5 < 35 \to 5 \pm 23 < 35 \pm 23$

    Sumando, $ 28 < 58$
    
    Restando, $ -18 < 12$

    \item $a = 15, b = 60, c = 75 \in R, 15 < 60 \to 15 \pm 75 < 60 \pm 75$

    Sumando, $ 90 < 135$
    
    Restando, $ -60 < -15$
\end{itemize}

\section{Multiplicación por número positivo}

$a, b, c \in R, a < b \;y\; c \in R^+ \to a\times c < b\times c$

\begin{itemize}
    \item $a = 6, b = 8, c = 2 \in R, 6 < 8 \;y\; c = 2 \to 6\times 2 < 8\times 2$
    
    $12 < 16$

    \item $a = -2, b = 6, c = 3 \in R, -2 < 6 \;y\; c = 3 \to -2\times 3 < 6\times 3$
    
    $-6 < 18$

    \item $a = 1, b = 2.5, c = 2 \in R, 1 < 2.5 \;y\; c = 2 \to 1\times 2 < 2.5\times 2$

    $2 < 5$
\end{itemize}

\section{Multiplicación por número negativo}

$a, b, c \in R, a < b \;y\; c \in R^- (c < 0) \to a\times c > b\times c$

\begin{itemize}
    \item $a = 6, b = 8, c = -2 \in R, 6 < 8 \;y\; c = -2 \to 6\times -2 > 8\times -2$
    
    $-12 < -16$

    \item $a = -2, b = 6, c = -3 \in R, -2 < 6 \;y\; c = -3 \to -2\times -3 > 6\times -3$
    
    $6 > -18$

    \item $a = 1, b = 2.5, c = -2 \in R, 1 < 2.5 \;y\; c = -2 \to 1\times -2 > 2.5\times -2$

    $-2 > -5$
\end{itemize}

\section{}

$a, b, c, d \in R, a < b \;y\; c < d \to a + c > b + d$

\begin{itemize}
    \item $a = 3, b = 5, c = 8, d = 10 \in R, 3 < 5 \;y\; 8 < 10 \to 3 + 8 < 5 + 1$
    
    $ 11 < 15$
\end{itemize}

\section{}

$Si \; a \times b > 0 \to a > 0 \;y\; b > 0 \; o \; a < 0 \;y\; b < 0$

\begin{itemize}
    \item $a = 3, b = 5, \;y\; 3 \times 5 > 0 \to 3 > 0 \;y\; 5 > 0$
    
    $15 > 0$

    \item $a = -3, b = -5, \;y\; -3 \times -5 > 0 \to -3 < 0 \;y\; -5 < 0$
    
    $15 > 0$
\end{itemize}

\section{}

$Si \; a \times b >= 0 \to a >= 0 \;y\; b >= 0 \; o \; a <= 0 \;y\; b <= 0$

\begin{itemize}
    \item $a = 3, b = 5, \;y\; 3 \times 5 >= 0 \to 3 >= 0 \;y\; 5 >= 0$
    
    $15 > 0$

    \item $a = -3, b = -5, \;y\; -3 \times -5 >= 0 \to -3 <= 0 \;y\; -5 <= 0$
    
    $15 > 0$

    \item $a = -3, b = 0, \;y\; -3 \times 0 >= 0 \to -3 <= 0 \;y\; 0 <= 0$
    
    $0 = 0$


\end{itemize}


\section*{ACTIVIDAD 2: INTERVALOS}

Traduzca cada conjunto en forma de intervalos y represéntalos en la recta.

\end{document}
